\documentclass{article}

\usepackage{arxiv}

\usepackage[utf8]{inputenc} % allow utf-8 input
\usepackage[T1]{fontenc}    % use 8-bit T1 fonts
\usepackage{lmodern}        % https://github.com/rstudio/rticles/issues/343
\usepackage{hyperref}       % hyperlinks
\usepackage{url}            % simple URL typesetting
\usepackage{booktabs}       % professional-quality tables
\usepackage{amsfonts}       % blackboard math symbols
\usepackage{nicefrac}       % compact symbols for 1/2, etc.
\usepackage{microtype}      % microtypography
\usepackage{graphicx}

\title{Infraestructura Critica}

\author{
    Grupo Messi
    \thanks{Ingeniero Ricardo Palma}
   \\
    Universidad Nacional de Cuyo \\
     \\
  Mendoza Argentina \\
  \texttt{} \\
  }


% tightlist command for lists without linebreak
\providecommand{\tightlist}{%
  \setlength{\itemsep}{0pt}\setlength{\parskip}{0pt}}



\begin{document}
\maketitle


\begin{abstract}
Las infraestructuras críticas son elementos esenciales y vulnerables
que, de ser perturbados, tendrían un impacto grave en los servicios
esenciales. La protección de estas infraestructuras es crucial para
prevenir ataques terroristas, lo que requiere medidas de protección
física, ciberseguridad y cooperación internacional. En el caso
específico del Puerto de Buenos Aires, se destaca la importancia de un
enfoque integral y personalizado para garantizar su seguridad ante
posibles amenazas.
\end{abstract}


\hypertarget{infraestructuras-cruxedticas}{%
\section{Infraestructuras Críticas}\label{infraestructuras-cruxedticas}}

Se entiende por infraestructuras críticas a las instalaciones
estratégicas cuyo funcionamiento es indispensable y no admiten
soluciones alternativas. Su perturbación, interrupción, interferencia o
destrucción tendría un grave impacto sobre los servicios esenciales.
Toda acción, deliberada o no, que atente contra el normal desempeño de
suministros como electricidad, agua, gas, Internet; transporte público
(trenes, autobuses de corta, media y larga distancia, metro/subte,
vuelos, medios fluviales o marítimos), cadena logística y de
distribución, etc. Supone eventos del mundo físico tales como
vandalismo, sabotaje, allanamientos, fenómenos meteorológicos,
terremotos, tsunamis, huracanes; o transacciones del universo virtual.
El quinto dominio representado por el ciberespacio se ha convertido en
un territorio donde se llevan a cabo batallas asimétricas en las que
desde un simple internauta hasta un poderoso Estado-Nación pueden
infligir daños de consideración a las infraestructuras críticas.

Las redes de infraestructura constituyen un elemento central de la
integración del sistema económico y territorial de los países.

Una infraestructura adecuada es un factor explicativo importante de la
capacidad de los países de diversificar sus economías, expandir el
comercio, responder al crecimiento demográfico, reducir la pobreza y
mejorar sus condiciones medioambientales.

Los debates respecto a la continuidad de los servicios de
infraestructura han adquirido mayor relevancia tras la eclosión de
combinaciones más complejas de peligros, y el aumento de la frecuencia y
magnitud de eventos extremos con grandes impactos sobre los sistemas de
transporte, energía, viviendas y servicios de infraestructura social. La
pandemia del COVID-19, por ejemplo, ha evidenciado la necesidad urgente
de garantizar que, en escenarios de crisis y cambios drásticos de
patrones de consumo, la infraestructura sea capaz de facilitar la
provisión fluida de servicios de transporte, conectividad y servicios
públicos.

No obstante ello, son relativamente pocos los países que los utilizan,
de forma sistemática, instrumentos para analizar y mitigar los riesgos
que se ciernen sobre la infraestructura.

\hypertarget{el-rol-de-la-infraestructura-en-el-desarrollo}{%
\section{El rol de la infraestructura en el
desarrollo}\label{el-rol-de-la-infraestructura-en-el-desarrollo}}

El rol fundamental que cumple la infraestructura en el proceso de
desarrollo ha sido ampliamente reconocido y analizado en la literatura.
Según el Besant-Jones y otros (1994), una infraestructura adecuada es un
factor explicativo importante de la capacidad de los países de
diversificar sus economías, expandir el comercio, responder al
crecimiento demográfico, reducir la pobreza y mejorar sus condiciones
medioambientales.

Las redes de infraestructura constituyen un elemento central de la
integración del sistema económico y territorial de los países.

Se argumenta que la provisión adecuada de infraestructura tiene efectos
positivos sobre la productividad de una economía y está asociada a la
reducción de los costos de producción.

Con respecto a los vínculos entre la oferta de infraestructura y el
crecimiento económico, a su vez, hay evidencias de una fuerte
correlación entre ambas variables, aunque no sean inequívocos el grado y
la dirección de la causalidad. Se asume, por lo tanto, que la dinámica
de la inversión en infraestructura y el crecimiento económico se
refuerzan mutuamente. Por otra parte, niveles insuficientes de inversión
en infraestructura son identificados como una de las principales causas
del bajo crecimiento económico en los países en desarrollo.

Como recalcan Fay y otros (2011), los vínculos entre infraestructura y
desarrollo no son, necesariamente, inmediatos y unívocos. Además de la
variedad de canales por medio de los cuales la oferta de infraestructura
económica se puede convertir en condiciones socioeconómicas favorables,
dichos efectos pueden variar significativamente entre países y a lo
largo del tiempo. Hay fuertes evidencias de que la calidad de la
infraestructura tiene rol importante en el proceso del desarrollo de las
condiciones socioeconómicas. Por ejemplo: es improbable que el impacto
social y económico de la implementación de una carretera de un solo
carril, sea idéntico al de una carretera con cinco carriles, aunque
tengan ambas la misma extensión.

Más allá de los factores económicos, la infraestructura tiene
implicaciones importantes en términos del desarrollo social, ya que
determina de forma directa el acceso de la población a servicios básicos
y asegura una mayor defensa contra desastres, naturales o provocados por
el humano. De modo indirecto, el aumento de la productividad de los
sectores de la economía, la reducción de los costos de transporte y la
creación de puestos de trabajo que pueden derivar de una mejor dotación
de servicios de infraestructura también pueden conducir a logros
sociales importantes. además de facilitar el acceso de los individuos
más pobres a oportunidades productivas y aumentar su capital humano por
medio del acceso a servicios de educación y salud, la infraestructura
cumple rol fundamental en la integración de esos individuos y sus
familias a la vida social y económica.

La infraestructura también tiene repercusiones importantes en el
medioambiente, ya que condiciona los patrones de consumo energético de
una economía, la generación de desechos y efluentes, y los niveles de
emisión de gases de efecto invernadero y otros contaminantes en la
atmósfera.

Aunque se reconozca el rol clave de la infraestructura en el desarrollo,
se ha documentado un cuadro generalizado de baja inversión en sistemas
de energía, transporte, telecomunicaciones, agua y saneamiento: en el
conjunto de economías avanzadas y emergentes, el stock de capital
público relativo al PIB ha disminuido en unos 15\% a lo largo de las
últimas tres décadas.

Para el caso de América Latina y el Caribe, la CEPAL ha desarrollado
estudios relacionados con el mismo tema en los últimos años. se ha
estimado la brecha de infraestructura para la región, calculada como la
diferencia entre la inversión en infraestructura y aquella necesaria
para satisfacer diversos objetivos de desarrollo. Se encontró que, para
el período de 2006 a 2020, sería necesario invertir anualmente en torno
al 6,2\% del PIB regional para atender a las necesidades de las empresas
y de los consumidores finales, al paso que, para alcanzar los niveles de
infraestructura per cápita de un conjunto de países del sudeste
asiático, las cifras anuales requeridas para igual período ascenderían
al 7,9\% del PIB. Estudios posteriores han demostrado que los niveles de
inversión observados en los países de la región han sido insuficientes
con relación a los valores recomendados por los autores, o si se los
compara con otras economías en desarrollo.

En el estudio más reciente que ha aplicado la misma metodología de la
brecha de infraestructura en América Latina y el Caribe (Sánchez y
otros, 2017). Las estimaciones actualizadas para la región

\hypertarget{caso-pruxe1ctico-1}{%
\section{Caso práctico 1}\label{caso-pruxe1ctico-1}}

Antes de adentrarnos en nuestro problema vamos a conocer un poco la
situación actual de las infraestructuras criticas de la base industrial
de defensa del ejercito argentino. El Ministerio de Defensa de la Nación
Argentina define las IC de la BID como ``aquellos activos o sistemas
físicos o virtuales que son esenciales para la producción, adquisición,
almacenamiento, distribución y utilización de los recursos materiales y
servicios necesarios para la defensa nacional''. Estas IC se
caracterizan por:

\begin{itemize}
\tightlist
\item
  Su criticidad: Son indispensables para el funcionamiento de la BID y
  la capacidad de defensa del país.
\item
  Su vulnerabilidad: Son susceptibles a ataques o eventos disruptivos
  que podrían afectar su funcionamiento.
\item
  Su interdependencia: Están interconectadas con otras IC, lo que
  amplifica el impacto de su afectación.
\end{itemize}

\hypertarget{normativas-que-rigen-las-ic-y-las-inextgen}{%
\subsection{Normativas que Rigen las IC y las
INEXTGEN}\label{normativas-que-rigen-las-ic-y-las-inextgen}}

En Argentina, existen diversas normativas que regulan las IC y las
INEXTGEN (infraestructuras críticas de nueva generación), incluyendo:

\begin{itemize}
\tightlist
\item
  Ley Nacional de Seguridad Cibernética (N° 27.341): Establece un marco
  legal para la protección de las IC frente a ciberataques.
\item
  Decreto N° 1000/2018: Aprueba la Estrategia Nacional de
  Ciberseguridad, que define las IC y establece lineamientos para su
  protección.
\item
  Resolución N° 115/2020: Establece los requisitos mínimos de seguridad
  para las IC.
\item
  Lineamientos de la Organización de Estados Americanos (OEA) para la
  Protección de Infraestructuras Críticas: Proporcionan recomendaciones
  para la identificación, evaluación, protección y recuperación de las
  IC.
\end{itemize}

\hypertarget{anuxe1lisis-de-las-ic-de-la-bid-en-el-contexto-regional-e-internacional}{%
\subsection{Análisis de las IC de la BID en el Contexto Regional e
Internacional}\label{anuxe1lisis-de-las-ic-de-la-bid-en-el-contexto-regional-e-internacional}}

Las IC de la BID del Ejército Argentino deben considerarse en el
contexto regional e internacional, donde existen diversas iniciativas y
organizaciones que trabajan en la protección de las IC. Algunas de estas
iniciativas son:

\begin{itemize}
\tightlist
\item
  El Comité Interamericano contra el Terrorismo (CICTE) de la OEA:
  Promueve la cooperación entre los países miembros para la protección
  de las IC.
\item
  El Centro Interamericano de Defensa (CID): Desarrolla estudios e
  investigaciones sobre las IC y la ciberdefensa.
\item
  La Organización del Tratado del Atlántico Norte (OTAN): Tiene un
  programa específico para la protección de las IC.
\end{itemize}

\hypertarget{quuxe9-ocurriruxeda-si-hay-un-ataque-terrorista-a-infraestructuras-cruxedticas-argentinas}{%
\subsection{¿Qué ocurriría si hay un ataque terrorista a
Infraestructuras Críticas
argentinas?}\label{quuxe9-ocurriruxeda-si-hay-un-ataque-terrorista-a-infraestructuras-cruxedticas-argentinas}}

El Ejército Argentino, como parte de las Fuerzas Armadas, tiene la
responsabilidad de contribuir a la defensa nacional y la seguridad
interior del país. En caso de un ataque terrorista a infraestructuras
críticas, el Ejército podría actuar de diversas maneras, de acuerdo con
las circunstancias específicas del ataque:

\begin{itemize}
\tightlist
\item
  Protección de las infraestructuras: El Ejército podría desplegar
  tropas para proteger las infraestructuras críticas y evitar nuevos
  ataques.
\item
  Restauración de servicios: El Ejército podría colaborar con otras
  entidades para restaurar los servicios afectados por el ataque.
\item
  Búsqueda y captura de los responsables: El Ejército podría participar
  en la búsqueda y captura de los responsables del ataque.
\item
  Mantenimiento del orden público: El Ejército podría colaborar con las
  fuerzas de seguridad para mantener el orden público y evitar
  disturbios sociales.
\item
  Cooperación internacional: El Ejército podría solicitar y/o brindar
  asistencia a otros países en la investigación y persecución de los
  responsables del ataque.
\end{itemize}

Es importante destacar que la respuesta del Ejército Argentino se
llevaría a cabo en coordinación con las autoridades civiles y de acuerdo
con las leyes vigentes. La decisión de utilizar las Fuerzas Armadas en
caso de un ataque terrorista sería tomada por las autoridades
competentes, teniendo en cuenta la gravedad de la situación y la
necesidad de proteger la seguridad nacional.

Además de la respuesta militar, un ataque terrorista requeriría una
respuesta integral que involucraría a diferentes sectores del gobierno,
incluyendo organismos de inteligencia (para identificar y prevenir
futuros ataques), fuerzas de seguridad (para investigar el ataque y
capturar a los responsables), entidades de emergencia (para atender a
las víctimas y restaurar los servicios afectados), organismos
gubernamentales (para coordinar la respuesta y tomar medidas para
mitigar el impacto del ataque) y por ultimo el sector privado (para
colaborar en la restauración de los servicios y la recuperación
económica).

\hypertarget{conclusiuxf3n}{%
\subsection{Conclusión}\label{conclusiuxf3n}}

La protección de las IC de la BID del Ejército Argentino es un desafío
complejo que requiere un enfoque integral y multisectorial. Es necesario
fortalecer el marco legal y regulatorio, desarrollar capacidades de
gestión de riesgos, implementar medidas de protección física y
cibernética, y fomentar la cooperación internacional. La comprensión de
las definiciones, características y normativas que rigen las IC y las
INEXTGEN en diversos sectores, tanto a nivel nacional como regional e
internacional, es fundamental para diseñar e implementar estrategias
efectivas de protección.

\hypertarget{caso-pruxe1ctico-2}{%
\section{Caso práctico 2}\label{caso-pruxe1ctico-2}}

\hypertarget{situaciuxf3n-de-problema}{%
\subsection{Situación de Problema}\label{situaciuxf3n-de-problema}}

El Puerto de Buenos Aires es una de las infraestructuras críticas más
importantes de Argentina, tanto por su volumen de actividad como por su
estratégica ubicación. En los últimos años, se ha incrementado la
importación de gas natural licuado (GNL) en barcos metaneros, lo que ha
generado preocupación por la posibilidad de que estos buques sean
objetivos de ataques terroristas.

\hypertarget{hipuxf3tesis}{%
\subsection{Hipótesis}\label{hipuxf3tesis}}

Existen diversos grupos terroristas con la capacidad y la motivación
para atacar objetivos como el Puerto de Buenos Aires y los barcos
metaneros. Esta preocupacion surge debido a que a partir de la
informacion recolectada se sabe que el promedio por barco es de 70
personas de origen asiatico o arabicos y gracias a la importacion de gas
licuado por barcos que deben llegar hasta el canal de Emilio Mitre para
depositar el gas. En caso de no seguir el camino correcto el barco
terminaria en el canal Norte llegando a Purto Nuevo, este barco llegaria
a una ciudad de Puerto Madero. Si se considera que la tripulación tiene
intenciones de hacer un ataque terrorista, solo se requiere de un mal
control de las coniciones de gas para que explote o ayuda de un dron que
uede estallar contra el barco en el momento que se encuentre en la
costa. El daño que ocasionaria una explosion como esta constaria de 7
kilometros a la redonda tomando desde la costa hasta el teatro Colón.

Un ataque de este tipo tendría graves consecuencias, incluyendo:

\begin{itemize}
\tightlist
\item
  Pérdidas de vidas humanas y daños a la propiedad.
\item
  Interrupción del suministro de gas natural, lo que podría provocar una
  crisis energética.
\item
  Daños a la economía nacional y la imagen internacional del país.
\end{itemize}

\hypertarget{tesis}{%
\subsection{Tesis}\label{tesis}}

Medidas para prevenir un ataque terrorista en el Puerto de Buenos Aires
y proteger los barcos metaneros, con énfasis en la infraestructura
crítica.

\begin{enumerate}
\def\labelenumi{\arabic{enumi}.}
\tightlist
\item
  Protección de la infraestructura crítica:
\end{enumerate}

1.1. Tanques de almacenamiento de gas:

Ubicación estratégica: Reubicar los tanques de almacenamiento de gas a
zonas más seguras dentro del puerto, alejados de áreas pobladas y de
fácil acceso. Protección física: Implementar barreras físicas robustas
alrededor de los tanques, como muros de hormigón armado o cercas de alta
seguridad. Sistemas de detección de intrusiones: Instalar sistemas de
detección de intrusiones para alertar al personal de seguridad en caso
de intentos de acceso no autorizado a los tanques. Sistemas de extinción
de incendios: Implementar sistemas de extinción de incendios avanzados y
de fácil activación para minimizar los daños en caso de un incendio o
explosión.

1.2. Tuberías: Monitoreo y control: Implementar sistemas de monitoreo y
control remoto de las tuberías para detectar fugas, manipulaciones o
anomalías en el flujo de gas. Protección contra daños: Instalar
protecciones físicas alrededor de las tuberías para evitar daños
accidentales o intencionales, como excavaciones o vandalismo. Válvulas
de cierre de emergencia: Instalar válvulas de cierre de emergencia en
puntos estratégicos de la red de tuberías para aislar secciones en caso
de una fuga o un ataque. Planes de mantenimiento preventivo: Implementar
planes de mantenimiento preventivo para detectar y corregir posibles
debilidades en las tuberías antes de que ocurran fallas.

1.3. Sistemas de control: Ciberseguridad: Fortalecer la seguridad
cibernética de los sistemas de control de la infraestructura crítica,
implementando medidas como autenticación multifactor, encriptación de
datos y segmentación de redes. Protección contra ataques cibernéticos:
Implementar sistemas de detección de intrusiones y de prevención de
intrusiones para proteger los sistemas de control contra ataques
cibernéticos. Copia de seguridad y recuperación de desastres:
Implementar planes de copia de seguridad y recuperación de desastres
para garantizar la continuidad operativa en caso de un ataque
cibernético o una falla del sistema. Capacitación del personal:
Capacitar al personal responsable de la operación y mantenimiento de los
sistemas de control en materia de ciberseguridad y protección contra
ataques cibernéticos.

\begin{enumerate}
\def\labelenumi{\arabic{enumi}.}
\setcounter{enumi}{1}
\item
  Integración de la seguridad de la infraestructura crítica en los
  planes de prevención: Evaluación de riesgos: Realizar evaluaciones de
  riesgos específicas para la infraestructura crítica del puerto,
  considerando las amenazas potenciales, las vulnerabilidades y las
  posibles consecuencias de un ataque. Desarrollo de planes de
  seguridad: Desarrollar planes de seguridad específicos para la
  infraestructura crítica, detallando las medidas de prevención,
  detección, respuesta y recuperación en caso de un ataque. Integración
  con los planes generales de seguridad: Integrar los planes de
  seguridad de la infraestructura crítica en los planes generales de
  seguridad del puerto, asegurando una respuesta coordinada y efectiva
  en caso de un ataque. Ejercicio y entrenamiento: Realizar ejercicios y
  simulacros de forma regular para probar los planes de seguridad de la
  infraestructura crítica y entrenar al personal en su implementación.
\item
  Tecnologías avanzadas para la protección de la infraestructura
  crítica: Sensores y detectores: Implementar sensores y detectores
  específicos para la infraestructura crítica, como sensores de gas,
  detectores de fugas, sensores de vibración y sistemas de detección de
  intrusiones. Monitoreo remoto y análisis de datos: Implementar
  sistemas de monitoreo remoto y análisis de datos para recopilar y
  analizar información en tiempo real sobre el estado de la
  infraestructura crítica, permitiendo identificar anomalías y detectar
  posibles amenazas. Sistemas de alerta temprana: Implementar sistemas
  de alerta temprana para notificar al personal de seguridad y a las
  autoridades en caso de detectar anomalías o posibles amenazas en la
  infraestructura crítica. Tecnologías de inteligencia artificial:
  Explorar el uso de tecnologías de inteligencia artificial para
  analizar datos, identificar patrones de comportamiento sospechoso y
  predecir posibles ataques a la infraestructura crítica.
\end{enumerate}

Es importante destacar que la selección e implementación de las medidas
de protección de la infraestructura crítica debe realizarse de manera
personalizada, considerando las características específicas del puerto,
las amenazas potenciales y los recursos disponibles.

La protección de la infraestructura crítica es un componente esencial de
la estrategia general de prevención de ataques terroristas en el Puerto
de Buenos Aires. Al implementar medidas efectivas para proteger los
tanques de almacenamiento de gas, las tuberías y los sistemas de
control, se puede reducir significativamente la vulnerabilidad del
puerto y minimizar las consecuencias de un posible ataque.

\hypertarget{conclusiuxf3n-1}{%
\subsection{Conclusión}\label{conclusiuxf3n-1}}

La posibilidad de un ataque terrorista en el Puerto de Buenos Aires
representa una amenaza significativa para la seguridad nacional, la
economía y la imagen del país. Es necesario adoptar un enfoque integral
para prevenir este tipo de ataques, que incluya medidas de seguridad
física, el uso de tecnologías avanzadas, la mejora de la coordinación y
la inteligencia, y la cooperación internacional.

La protección de la infraestructura crítica es un elemento fundamental
de este enfoque. Los tanques de almacenamiento de gas, las tuberías y
los sistemas de control son objetivos potenciales para los terroristas,
y su vulnerabilidad podría tener consecuencias devastadoras. Es
necesario implementar medidas específicas para proteger esta
infraestructura, como barreras físicas, sistemas de detección de
intrusiones, sistemas de extinción de incendios y planes de
mantenimiento preventivo.

Las tecnologías avanzadas también juegan un papel importante en la
prevención de ataques terroristas. Sensores y detectores, sistemas de
monitoreo remoto, análisis de datos, sistemas de alerta temprana e
inteligencia artificial pueden ser herramientas valiosas para
identificar anomalías, detectar amenazas y prevenir ataques antes de que
ocurran.

La coordinación y la inteligencia son esenciales para una respuesta
efectiva a los ataques terroristas. Es necesario establecer canales de
comunicación fluidos entre las fuerzas de seguridad, las agencias de
inteligencia, las autoridades portuarias y otras entidades relevantes.
Se debe compartir información de manera oportuna y se deben desarrollar
planes de respuesta coordinados para minimizar los daños y salvar vidas.

La cooperación internacional es otro componente clave de la estrategia
de prevención. Es necesario colaborar con países vecinos y
organizaciones internacionales para compartir información de
inteligencia, realizar operaciones conjuntas de seguridad y desarrollar
estrategias de prevención a nivel regional.

En definitiva, la prevención de ataques terroristas en el Puerto de
Buenos Aires es una responsabilidad compartida que requiere el
compromiso de todos los actores involucrados. El gobierno, las fuerzas
de seguridad, el sector privado y la población en general deben trabajar
juntos para implementar las medidas necesarias y garantizar la seguridad
de este importante puerto.

Es importante destacar que este informe solo presenta un marco general
para la prevención de ataques terroristas en el Puerto de Buenos Aires.
La implementación de medidas específicas debe realizarse de manera
personalizada, considerando las características específicas del puerto,
las amenazas potenciales y los recursos disponibles.

\bibliographystyle{unsrt}
\bibliography{references.bib}


\end{document}
